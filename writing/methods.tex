%methods 
We chose to write our algorithms in C++ using OpenCV libraries. Since C++ and OpenCV are available on both Android and iOS platforms as well, the code should be relatively easy to port to those platforms. To allow for faster testing of our code, we develop the code on OS X.

Eclipse is used to compile each iteration of the algorithm into a debug-mode executable program named Recolor. For ease of testing, as the algorithm is modified, we add more functionality to the Recolor program and retain the ability to use previous versions of the algorithm. We use a custom Python script to run new versions of Recolor from the terminal to test it. All of the relevant code and its versions are hosted on a git repository at github.com/tiantianhan/recolor

Recolor takes as input a hand image, a mask instructing it where to find the average skin colour of the hand, and a desired target skin colour. (Other flags and inputs are also used for testing purposes, see the Github repository README file for a full description of the usage.) Recolor then outputs the processed image where the skin tone is adjusted to the target colour.

We iterated from simple to more complex algorithms, at each step testing the algorithm and evaluating the results. We tested progressive iterations on a set of hand images with varying skin tones. The images are are shown in Figure \ref{img:input_hands_1}.

%all images that are not test results will be copied to the images folder

\begin{figure}[H]
    \centering
    \begin{subfigure}[b]{0.20\textwidth}
        \includegraphics[width=\textwidth]{images/hand_dark}
        \caption{}\label{img:input_hands_1_dark}
    \end{subfigure}
    ~
    \begin{subfigure}[b]{0.20\textwidth}
        \includegraphics[width=\textwidth]{images/hand_brown}
        \caption{}\label{img:input_hands_1_brown}
    \end{subfigure}
    ~
    \begin{subfigure}[b]{0.20\textwidth}
        \includegraphics[width=\textwidth]{images/hand_light}
        \caption{}\label{img:input_hands_1_light}
    \end{subfigure}
    ~
    \begin{subfigure}[b]{0.20\textwidth}
        \includegraphics[width=\textwidth]{images/hand_pale}
        \caption{}\label{img:input_hands_1_pale}
    \end{subfigure}
    \caption{Different hand images used for testing}\label{img:input_hands_1}
\end{figure}

 For each test, we called the Recolor program to transform the image of one hand to have the skin tone of the hand in another image, then visually compared the processed image to the image of the target hand. We performed the process on all possible combinations of our test images, paying particular attention to the extreme cases, such as transforming from Figure \ref{img:input_hands_1_dark} to Figure \ref{img:input_hands_1_pale} and vice versa, as well as cases that start with a hand with mid-tone skin such as in Figure \ref{img:input_hands_1_brown} (as this is the most likely use case for applications that change the image of a model's hand to match a range of skin tones). We evaluated the resulting images subjectively, based on whether the processed hand looks believably like a hand naturally of that skin tone, and noted any flaws that we then attempted to correct with the next iteration of the algorithm.

In the following subsections we summarize the results of each algorithm and our evaluation of the results.

\subsection{Naive approach: simple $RGB$ colour addition}
%boost algo
\subsubsection*{Algorithm}
To begin, we performed a simple addition of a value to each of the $rgb$ channels of the hand, such that the average colour of the hand in the processed image is equal to the average colour of the hand in the target image. The algorithm is shown in Equation \ref{eq:boost_algo}.

\begin{equation} \label{eq:boost_algo}
r' = r + \delta_r
\end{equation}

Where 

\begin{equation*}
\delta_r = \mean{r_t} - \mean{r}
\end{equation*}

\nomenclature{$r$}{Value of the red channel of a pixel in the original hand image.}
\nomenclature{$r'$}{Value of the red channel of the pixel after an algorithm is applied.}
\nomenclature{$\mean{r}$}{Average value of the red channel of the original hand image.}
\nomenclature{$\mean{r_t}$}{Average value of the red channel of the target hand image.}
\nomenclature{$g$}{Value of the green channel of a pixel in the original hand image.}
\nomenclature{$b$}{Value of the blue channel of a pixel in the original hand image.}

With the same equation applying for the $g$ and $b$ channels.

\subsubsection*{Results}
The complete results are shown in Table \ref{tab:boost_test} in Appendix \ref{app:boost}, a portion is shown here for convenience.

\begin{longtable}{|c||c|c|c|}
    \caption*{Portion of test results of simple addition / subtraction brightening function from Table \ref{tab:boost_test} in the Appendix \ref{app:boost}}\\
    \hline
    No. & Original & Target & Results \\ 
      \hline  \ref{row:boost_test_hand_dark_to_hand_light} &
  \begin{minipage}{.29\textwidth}
    \includegraphics[width=\textwidth,height=\textheight,keepaspectratio]{../inputs/hand_dark.jpg}
  \end{minipage} & 
  \begin{minipage}{.29\textwidth}
    \includegraphics[width=\textwidth,height=\textheight,keepaspectratio]{../inputs/hand_light.jpg}
  \end{minipage} & 
  \begin{minipage}{.29\textwidth}
    \includegraphics[width=\textwidth,height=\textheight,keepaspectratio]{../rc_test/outputs/20170516_boost_test/hand_dark_to_hand_light.jpg}
  \end{minipage} \\
    \hline  \ref{row:boost_test_hand_brown_to_hand_dark} &
  \begin{minipage}{.29\textwidth}
    \includegraphics[width=\textwidth,height=\textheight,keepaspectratio]{../inputs/hand_brown.jpg}
  \end{minipage} & 
  \begin{minipage}{.29\textwidth}
    \includegraphics[width=\textwidth,height=\textheight,keepaspectratio]{../inputs/hand_dark.jpg}
  \end{minipage} & 
  \begin{minipage}{.29\textwidth}
    \includegraphics[width=\textwidth,height=\textheight,keepaspectratio]{../rc_test/outputs/20170516_boost_test/hand_brown_to_hand_dark.jpg}
  \end{minipage} \\
\hline  \ref{row:boost_test_hand_brown_to_hand_light} &
  \begin{minipage}{.29\textwidth}
    \includegraphics[width=\textwidth,height=\textheight,keepaspectratio]{../inputs/hand_brown.jpg}
  \end{minipage} & 
  \begin{minipage}{.29\textwidth}
    \includegraphics[width=\textwidth,height=\textheight,keepaspectratio]{../inputs/hand_light.jpg}
  \end{minipage} & 
  \begin{minipage}{.29\textwidth}
    \includegraphics[width=\textwidth,height=\textheight,keepaspectratio]{../rc_test/outputs/20170516_boost_test/hand_brown_to_hand_light.jpg}
  \end{minipage} \\
  \hline  \ref{row:boost_test_hand_brown_to_hand_pale} &
  \begin{minipage}{.29\textwidth}
    \includegraphics[width=\textwidth,height=\textheight,keepaspectratio]{../inputs/hand_brown.jpg}
  \end{minipage} & 
  \begin{minipage}{.29\textwidth}
    \includegraphics[width=\textwidth,height=\textheight,keepaspectratio]{../inputs/hand_pale.jpg}
  \end{minipage} & 
  \begin{minipage}{.29\textwidth}
    \includegraphics[width=\textwidth,height=\textheight,keepaspectratio]{../rc_test/outputs/20170516_boost_test/hand_brown_to_hand_pale.jpg}
  \end{minipage} \\
    \hline
\end{longtable}

\subsubsection*{Evaluation}
Images of darker skin tones and smaller changes between the original skin tone and target colour to begin with (Row \ref{row:boost_test_hand_brown_to_hand_dark}) tend to have better results than images with large changes, especially towards lighter colours. This is likely because large changes force bright points in the original image to be truncated at white, and also causes dark regions on the image, such as shadows and grooves, to become significantly brighter and less close to true black, giving the image a ``high-key" look (Row \ref{row:boost_test_hand_dark_to_hand_light} and \ref{row:boost_test_hand_brown_to_hand_light}).

In addition, we noted that at this stage the transformation from a dark coloured hand to a very pale hand, or even from a mid-toned hand to a pale hand and vice versa is especially unconvincing. (Row \ref{row:boost_test_hand_brown_to_hand_pale}, also see \ref{row:boost_test_hand_dark_to_hand_pale} and \ref{row:boost_test_hand_pale_to_hand_dark})


\subsection{Proportional adjustment relative to average colour} \label{sec:algo_prop_eval}
%prop algorithm methods
\subsubsection*{Algorithm}
To correct for the effect of the bright spots in the image being over bright and the high-key appearance resulting from all the shadows being brightened, we used an algorithm that maps the black and white points of the image to the same value, and adjusts the colours in between to match the target average colour. The algorithm is shown in Equation \ref{eq:prop_algo}.

\begin{equation} \label{eq:prop_algo}
  r' = \left.
  \begin{dcases}
    \displaystyle \Big(\frac{\mean{r_t}}{\mean{r}}\Big)r, & \text{for } r \leq \mean{r} \\
    \displaystyle 255 - 
    \Big(\frac{255 - \mean{r_t}}{255 - \mean{r}}\Big)(255 - r), & \text{for } r > \mean{r} \\
  \end{dcases}
  \right.
\end{equation}



With the same equation applying for the $g$ and $b$ channels.

\subsubsection*{Results}
The complete results are shown in Table \ref{tab:prop_test} in Appendix \ref{app:prop}, and a portion is shown here for convenience.

\begin{longtable}{|c||c|c|c|}
    \caption*{Portion of test results of adjusting proportionally based on distance of col or to the average from Table \ref{tab:prop_test} in the Appendix \ref{app:prop}}\\
    \hline
    No. & Original & Target & Results \\
    \hline  \ref{row:prop_test_hand_dark_to_hand_light} &
  \begin{minipage}{.29\textwidth}
    \includegraphics[width=\textwidth,height=\textheight,keepaspectratio]{../inputs/hand_dark.jpg}
  \end{minipage} & 
  \begin{minipage}{.29\textwidth}
    \includegraphics[width=\textwidth,height=\textheight,keepaspectratio]{../inputs/hand_light.jpg}
  \end{minipage} & 
  \begin{minipage}{.29\textwidth}
    \includegraphics[width=\textwidth,height=\textheight,keepaspectratio]{../rc_test/outputs/20170516_proportional_test/hand_dark_to_hand_light.jpg}
  \end{minipage} \\
    \hline  \ref{row:prop_test_hand_brown_to_hand_dark} &
  \begin{minipage}{.29\textwidth}
    \includegraphics[width=\textwidth,height=\textheight,keepaspectratio]{../inputs/hand_brown.jpg}
  \end{minipage} & 
  \begin{minipage}{.29\textwidth}
    \includegraphics[width=\textwidth,height=\textheight,keepaspectratio]{../inputs/hand_dark.jpg}
  \end{minipage} & 
  \begin{minipage}{.29\textwidth}
    \includegraphics[width=\textwidth,height=\textheight,keepaspectratio]{../rc_test/outputs/20170516_proportional_test/hand_brown_to_hand_dark.jpg}
  \end{minipage} \\
\hline  \ref{row:prop_test_hand_brown_to_hand_light} &
  \begin{minipage}{.29\textwidth}
    \includegraphics[width=\textwidth,height=\textheight,keepaspectratio]{../inputs/hand_brown.jpg}
  \end{minipage} & 
  \begin{minipage}{.29\textwidth}
    \includegraphics[width=\textwidth,height=\textheight,keepaspectratio]{../inputs/hand_light.jpg}
  \end{minipage} & 
  \begin{minipage}{.29\textwidth}
    \includegraphics[width=\textwidth,height=\textheight,keepaspectratio]{../rc_test/outputs/20170516_proportional_test/hand_brown_to_hand_light.jpg}
  \end{minipage} \\
  \hline  \ref{row:prop_test_hand_brown_to_hand_pale} &
  \begin{minipage}{.29\textwidth}
    \includegraphics[width=\textwidth,height=\textheight,keepaspectratio]{../inputs/hand_brown.jpg}
  \end{minipage} & 
  \begin{minipage}{.29\textwidth}
    \includegraphics[width=\textwidth,height=\textheight,keepaspectratio]{../inputs/hand_pale.jpg}
  \end{minipage} & 
  \begin{minipage}{.29\textwidth}
    \includegraphics[width=\textwidth,height=\textheight,keepaspectratio]{../rc_test/outputs/20170516_boost_test/hand_brown_to_hand_pale.jpg}
  \end{minipage} \\
    \hline
\end{longtable}

\subsubsection*{Evaluation}
This method improved the appearance of cases with over-bright spots or ``high-key" appearance issues, as Figure \ref{img:compare_bright_spot} shows:
\begin{figure}[H]
    \centering
    \begin{subfigure}[b]{0.40\textwidth}
        \includegraphics[width=\textwidth]{../rc_test/outputs/20170516_boost_test/hand_dark_to_hand_light.jpg}
        \caption{Simple addition algorithm (Equation \ref{eq:boost_algo})result}
    \end{subfigure}
    ~
    \begin{subfigure}[b]{0.40\textwidth}
        \includegraphics[width=\textwidth]{../rc_test/outputs/20170516_proportional_test/hand_dark_to_hand_light.jpg}
        \caption{Proportional adjustment algorithm (Equation \ref{eq:prop_algo}) result}
    \end{subfigure}
    \caption{Comparison of algorithm \ref{eq:boost_algo} and \ref{eq:prop_algo} results for transforming a dark hand (Figure \ref{img:input_hands_1_dark}) to a light hand (Figure \ref{img:input_hands_1_light}).\label{img:compare_bright_spot}}
\end{figure}

We noted however, that this method noticeably does not correct for, and even exacerbates slightly relative to the simple addition algorithm the dark spots at the joints and creases of a hand of darker skin tone when it is transformed to a lighter skin tone (Row \ref{row:prop_test_hand_brown_to_hand_light}). Other results are similar to the results of the simple addition algorithm.
 

\subsection{Correcting for dark spots}
%proportional corrected algorithm methods
\subsubsection*{Algorithm}
We attempted to correct the dark spot issue by significantly reducing the absolute difference between dark pixels and the average colour, ensuring that the dark spots would instead have colours close to the average. We perform this correction on the output of the proportional adjustment algorithm.

\begin{equation} \label{eq:prop_corr_algo}
  r'' = \left.
  \begin{dcases}
    \displaystyle \mean{r'} - \frac{(\mean{r'} - r')}{\alpha}, & \text{for } r' < \mean{r'} \\
    \displaystyle r', & \text{for } r' \geq \mean{r'} \\
  \end{dcases}
  \right.\\
\end{equation}

\nomenclature{$r''$}{Value of the red channel of the pixel after a second algorithm is applied.}
\nomenclature{$\mean{r'}$}{Average value of the red channel of the hand image outputed by the first algorithm.}

Where $\alpha$ is a constant, $\alpha  > 1$. The same equation applies for the $g$ and $b$ channels.

\subsubsection*{Results}
See Table \ref{tab:prop_correct_test} in Appendix \ref{app:prop_corr_a10}, \ref{app:prop_corr_a5} and \ref{app:prop_corr_a3} for complete results for a range of values for $\alpha$. A portion of the results for $\alpha = 3$ is reproduced here for convenience.

\begin{longtable}{|c||c|c|c|}
    \caption*{Test results of proportional adjusting with correction for dark spots, alpha = 3 from Table \ref{tab:prop_correct_test_a3} in the Appendix \ref{app:prop_corr_a3}}\\
    \hline
    No. & Original & Target & Results \\
    \hline  \ref{row:prop_correct_test_a3_hand_dark_to_hand_brown} &
  \begin{minipage}{.29\textwidth}
    \includegraphics[width=\textwidth,height=\textheight,keepaspectratio]{../inputs/hand_dark.jpg}
  \end{minipage} & 
  \begin{minipage}{.29\textwidth}
    \includegraphics[width=\textwidth,height=\textheight,keepaspectratio]{../inputs/hand_light.jpg}
  \end{minipage} & 
  \begin{minipage}{.29\textwidth}
    \includegraphics[width=\textwidth,height=\textheight,keepaspectratio]{../rc_test/outputs/20170517_proportional_corrected_test_alpha3/hand_dark_to_hand_light.jpg}
  \end{minipage} \\
    \hline  \ref{row:prop_correct_test_a3_hand_brown_to_hand_dark} &
  \begin{minipage}{.29\textwidth}
    \includegraphics[width=\textwidth,height=\textheight,keepaspectratio]{../inputs/hand_brown.jpg}
  \end{minipage} & 
  \begin{minipage}{.29\textwidth}
    \includegraphics[width=\textwidth,height=\textheight,keepaspectratio]{../inputs/hand_dark.jpg}
  \end{minipage} & 
  \begin{minipage}{.29\textwidth}
    \includegraphics[width=\textwidth,height=\textheight,keepaspectratio]{../rc_test/outputs/20170517_proportional_corrected_test_alpha3/hand_brown_to_hand_dark.jpg}
  \end{minipage} \\
\hline  \ref{row:prop_correct_test_a3_hand_brown_to_hand_light} &
  \begin{minipage}{.29\textwidth}
    \includegraphics[width=\textwidth,height=\textheight,keepaspectratio]{../inputs/hand_brown.jpg}
  \end{minipage} & 
  \begin{minipage}{.29\textwidth}
    \includegraphics[width=\textwidth,height=\textheight,keepaspectratio]{../inputs/hand_light.jpg}
  \end{minipage} & 
  \begin{minipage}{.29\textwidth}
    \includegraphics[width=\textwidth,height=\textheight,keepaspectratio]{../rc_test/outputs/20170517_proportional_corrected_test_alpha3/hand_brown_to_hand_light.jpg}
  \end{minipage} \\
  \hline  \ref{row:prop_correct_test_a3_hand_brown_to_hand_pale} &
  \begin{minipage}{.29\textwidth}
    \includegraphics[width=\textwidth,height=\textheight,keepaspectratio]{../inputs/hand_brown.jpg}
  \end{minipage} & 
  \begin{minipage}{.29\textwidth}
    \includegraphics[width=\textwidth,height=\textheight,keepaspectratio]{../inputs/hand_pale.jpg}
  \end{minipage} & 
  \begin{minipage}{.29\textwidth}
    \includegraphics[width=\textwidth,height=\textheight,keepaspectratio]{../rc_test/outputs/20170517_proportional_corrected_test_alpha3/hand_brown_to_hand_pale.jpg}
  \end{minipage} \\
    \hline
\end{longtable}

\subsubsection*{Evaluation}
As show in Figure \ref{img:compare_dark_spot}, the dark spots and creases noted in Section \ref{sec:algo_prop_eval} are reduced.

\begin{figure}[H]
    \centering
    \begin{subfigure}[b]{0.40\textwidth}
        \includegraphics[width=\textwidth]{../rc_test/outputs/20170516_proportional_test/hand_brown_to_hand_light.jpg}
        \caption{Proportional adjustment algorithm (Equation \ref{eq:prop_algo})result}
    \end{subfigure}
    ~
    \begin{subfigure}[b]{0.40\textwidth}
        \includegraphics[width=\textwidth]{../rc_test/outputs/20170517_proportional_corrected_test_alpha3/hand_brown_to_hand_light.jpg}
        \caption{Proportional adjustment algorithm with correction (Equation \ref{eq:prop_corr_algo}) result}
    \end{subfigure}
    \caption{Comparison of algorithms \ref{eq:prop_algo} and \ref{eq:prop_corr_algo} results for transforming a mid-toned hand (Figure \ref{img:input_hands_1_brown}) to a light hand (Figure \ref{img:input_hands_1_light}).\label{img:compare_dark_spot}}
\end{figure}

However, we have noted that even for the more lower strength effect where $\alpha = 3$, this process brings back the problem of eliminating true black from the image, making the image appear once again to be without shadows (See Row \ref{row:prop_correct_test_a3_hand_dark_to_hand_brown} and \ref{row:prop_correct_test_a3_hand_brown_to_hand_light}). 

Also, it should be noted that for completeness, we performed this algorithm on all possible combinations of hand images, even though it was targeted at cases where the original image is of a skin tone darker than the new image. 

Interestingly, in a couple of cases of transformation from lighter skin tone to darker skin tone, the hand takes on a darker colour around joints in a similar manner to how a hand of dark skin tone is naturally coloured, though the algorithm was not intended for this purpose (See Row \ref{row:prop_correct_test_a3_hand_brown_to_hand_dark} as well as \ref{row:prop_correct_test_a3_hand_light_to_hand_dark}, \ref{row:prop_correct_test_a3_hand_light_to_hand_brown}). This is not always the case, however. Row \ref{row:prop_correct_test_a3_hand_pale_to_hand_dark} does not seem to exhibit this effect, possibly due to the different lighting of the hand in the image.

\subsection{Ignoring shadows in calculating average colour}
%adding correction for percentage of skin
\subsubsection*{Algorithm}
We noticed that in the case where the target image has a pale hand and relatively dark shadows such as in Figure \ref{img:input_hands_1_light}, the average colour calculated for the target hand is too dark, causing the skin colour of the results to appear darker than the skin colour of the target. We correct this by calculating the average skin colour of the target hand with only a percentage of the brightest pixels in the original region of interest used to calculate the average. 

\begin{equation} \label{eq:prop_corr_ave_algo}
  %TODO add equation
\end{equation}

\subsubsection*{Results}
We generated the resultant images using several different values for the percentile of brightest pixels, and determined that the 10th percentile gave the best results, as shown in Table \ref{tab:prop_correct_test_a1p1_ave10}. In Appendices \ref{app:prop_corr_ave_a1p1_perc5} and \ref{app:prop_corr_ave_a1p1_perc25} we show the results for some other percentiles.

\begin{longtable}{|N||c|c|c|}
	\caption{Test results of proportional brightening with correction for dark spots and using brightest 10 percent of pixels to calculate average colour, $\alpha$ = 1.1\label{tab:prop_correct_test_a1p1_ave10}}\\
	\hline
	\multicolumn{1}{|c||}{No.} & Original & Target & Result \\ 
	\hline
	\input{latex_maker/prop_correct_ave_test_a1p1_perc10_label/hand_dark_to_hand_brown}
	  \label{row:PY_NAME_hand_dark_to_hand_light} &
  \begin{minipage}{.29\textwidth}
    \includegraphics[width=\textwidth,height=\textheight,keepaspectratio]{../inputs/hand_dark.jpg}
  \end{minipage} & 
  \begin{minipage}{.29\textwidth}
    \includegraphics[width=\textwidth,height=\textheight,keepaspectratio]{../inputs/hand_light.jpg}
  \end{minipage} & 
  \begin{minipage}{.29\textwidth}
    \includegraphics[width=\textwidth,height=\textheight,keepaspectratio]{../rc_test/outputs/20170524_prop_corr_1p1_ave_100/hand_dark_to_hand_light.jpg}
  \end{minipage} \\
\hline
	\input{latex_maker/prop_correct_ave_test_a1p1_perc10_label/hand_dark_to_hand_pale}
	\input{latex_maker/prop_correct_ave_test_a1p1_perc10_label/hand_brown_to_hand_dark}
	\input{latex_maker/prop_correct_ave_test_a1p1_perc10_label/hand_brown_to_hand_light}
	\input{latex_maker/prop_correct_ave_test_a1p1_perc10_label/hand_brown_to_hand_pale}
	\input{latex_maker/prop_correct_ave_test_a1p1_perc10_label/hand_light_to_hand_dark}
	\input{latex_maker/prop_correct_ave_test_a1p1_perc10_label/hand_light_to_hand_brown}
	  \label{row:PY_NAME_hand_light_to_hand_pale} &
  \begin{minipage}{.29\textwidth}
    \includegraphics[width=\textwidth,height=\textheight,keepaspectratio]{../inputs/hand_light.jpg}
  \end{minipage} & 
  \begin{minipage}{.29\textwidth}
    \includegraphics[width=\textwidth,height=\textheight,keepaspectratio]{../inputs/hand_pale.jpg}
  \end{minipage} & 
  \begin{minipage}{.29\textwidth}
    \includegraphics[width=\textwidth,height=\textheight,keepaspectratio]{../rc_test/outputs/20170524_prop_corr_1p1_ave_100/hand_light_to_hand_pale.jpg}
  \end{minipage} \\
\hline
	  \ref{row:PY_NAME_hand_pale_to_hand_dark} &
  \begin{minipage}{.29\textwidth}
    \includegraphics[width=\textwidth,height=\textheight,keepaspectratio]{../inputs/hand_pale.jpg}
  \end{minipage} & 
  \begin{minipage}{.29\textwidth}
    \includegraphics[width=\textwidth,height=\textheight,keepaspectratio]{../inputs/hand_dark.jpg}
  \end{minipage} & 
  \begin{minipage}{.29\textwidth}
    \includegraphics[width=\textwidth,height=\textheight,keepaspectratio]{../rc_test/outputs/20170524_prop_corr_1p1_ave_25/hand_pale_to_hand_dark.jpg}
  \end{minipage} \\
\hline
	\input{latex_maker/prop_correct_ave_test_a1p1_perc10_label/hand_pale_to_hand_brown}
	  \label{row:PY_NAME_hand_pale_to_hand_light} &
  \begin{minipage}{.29\textwidth}
    \includegraphics[width=\textwidth,height=\textheight,keepaspectratio]{../inputs/hand_pale.jpg}
  \end{minipage} & 
  \begin{minipage}{.29\textwidth}
    \includegraphics[width=\textwidth,height=\textheight,keepaspectratio]{../inputs/hand_light.jpg}
  \end{minipage} & 
  \begin{minipage}{.29\textwidth}
    \includegraphics[width=\textwidth,height=\textheight,keepaspectratio]{../rc_test/outputs/20170524_prop_corr_1p1_ave_10/hand_pale_to_hand_light.jpg}
  \end{minipage} \\
\hline

 \end{longtable}

\subsubsection*{Evaluation}
Figure \ref{img:10_perc_mask} demonstrates the new regions used to calculate the average skin colour. We can see that the areas with shadows are effectively discarded, and the average colour calculated is significantly lighter and visually closer to the skin colour of the target hand. 

\begin{figure}[H]
\centering
\begin{tabular}{ccc}
    \multirow{2}{*}[5em]{\begin{subfigure}[b]{0.30\textwidth}
        \includegraphics[width=\textwidth]{images/hand_pale}
        \caption{Input hand image with significant shadows}\label{img:alg_3_eval_hand_pale}
    \end{subfigure}}&
    \begin{subfigure}[b]{0.30\textwidth}
        \includegraphics[width=\textwidth]{images/pale_ave_10_original_mask}
        \caption{Mask used to calculate color in Algorithm \ref{eq:prop_corr_algo}}\label{img:original_mask}
    \end{subfigure} &
    \begin{subfigure}[b]{0.30\textwidth}
        \includegraphics[width=\textwidth]{images/ave_col_100}
        \caption{Average color calculated with Algorithm \ref{eq:prop_corr_algo}}\label{img:ave_col_100}
    \end{subfigure}\\
    &
    \begin{subfigure}[b]{0.30\textwidth}
        \includegraphics[width=\textwidth]{images/pale_ave_10_adjusted_mask}
        \caption{Mask used to calculate color in Algorithm \ref{eq:prop_corr_ave_algo}}\label{img:adjusted_mask}
    \end{subfigure} &
    \begin{subfigure}[b]{0.30\textwidth}
        \includegraphics[width=\textwidth]{images/ave_col_10}
        \caption{Average color calculated with Algorithm \ref{eq:prop_corr_ave_algo}}\label{img:ave_col_10}
    \end{subfigure}
\end{tabular}
\caption{The average color calculation process in Algorithm \ref{eq:prop_corr_ave_algo} compared the previous version, Algorithm \ref{eq:prop_corr_algo}}\label{img:10_perc_mask}
\end{figure}
\pagebreak

\subsection{Summary and evaluation of the complete algorithm}
%Summary of the complete algorithm
The complete algorithm (comprised of Algorithms \ref{eq:prop_algo}, \ref{eq:prop_corr_algo} and \ref{eq:prop_corr_ave_algo}) is summarized in the flow chart in Figure \ref{img:algo_summary}.

\begin{figure}[H]
    \centering
    \includegraphics[width=\textwidth]{images/algo_diagram}
    \caption{Summary of complete algorithm for hand colour transfer}\label{img:algo_summary}
\end{figure}

In reference to our original goals and constraints listed in Section \ref{sec:goals}, we have made the most progress in the area of accurate and realistic transfer of skin colour from mid-toned and light coloured hands to other colours. However, the range of colours that we can realistic transfer between remains limited. For future work, we should further test our algorithm on a larger set of hand images to determine whether there are any other issues. Finally, we have yet to profile and optimize the code for performance on a mobile devices; however, since we have written the code using a language and libraries that should be easy to port to a mobile platform, we believe that optimizing and porting the code should be a very feasible task for future work.