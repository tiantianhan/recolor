%introduction
We often cannot judge based on the appearance of a product worn by a model how that product will actually look on ourselves. Compared to users, models may have different skin colour and other bodily features. Ideally, to help the user make a better informed purchasing decision, the model demonstrating the product should be customized to each user. To achieve this, there are already many applications developed for the virtual trying on of products in the beauty, cosmetics and garment industries which digitally modify the images of models to take on the appearance of the user \cite{zhang_2017_try} \cite{shilkrot_2013_garment, li_2015_replace}.

For this project, we have a video in a mobile app demonstrating nail polish on a model hand to allow virtual try-on of different nail polish colours. We would like to edit this demo video so that the model appears to have the user's exact skin colour, to help the user better determine whether the nail polish colours look pleasant on the user's own hand. While it is possible to manually prepare a series of demo videos with models of different skin colours, preparing even a single video for virtual try-on is an extremely time intensive task. Moreover, each person has a particular skin colour and it's preferable to be able to tailor the video exactly to the skin colour of the user while the user is using the app.

To address these challenges, we propose developing an algorithm to incorporate into the app that quickly and automatically performs the image editing task. The user should be able to provide an image of their own hand as input, and the resulting edited model hand should be convincing and accurate to the skin colour of the provided user image. A wide range of user skin colours should be supported by a single model of mid-toned skin. Finally, the process should be quick to run on a mobile device, such that the user notices no significant time lag to see the resulting video frames upon inputting their own skin colour.

Currently, we aren't aware of an existing algorithm that satisfies all our specific requirements. While there has been a large body of work done addressing transfer of colour between images in general \cite{reinhard_2001_transfer, pitie_2005_pdf, chen_2014_propagation, chang_2015_palette, zhang_2017_decomposition}, only a smaller set of work specifically addresses transfer of skin colour \cite{yin_2004_transfer, seo_2005_transfer, yang_2017_semantic}. All such studies address face skin colour rather than hand skin colour, which often means that more of the study is devoted to handling colour transfer of different, more complex aspects of the face \cite{yang_2017_semantic}. Skin colour transfer of the whole body are sometimes performed in other, more general imaging processing applications, but in those cases, since the skin colour transfer is often only a small part of the whole process, the algorithm used is often relatively simple and not heavily designed for achieving accuracy to the user skin colour \cite{shilkrot_2013_garment, li_2015_replace}. In the related field of skin colour enhancement applications, the methods used often are not meant to make large changes to the user skin colour \cite{aradhye_2009_enhancement, lee_2010_mobile}. Finally, algorithms developed by most of the prior studies do not appear to be meant for use with the limited resources on a mobile device.

Our goal is to develop a mobile compatible recolouring algorithm that would satisfy our requirements. As sub-objectives, we would like to first develop an effective algorithm and then optimize the algorithm's running time. We will focus solely on achieving convincing colour transfer in the algorithm, and assume that the location of the skin in the images are already determined by an another process. 

For developing and testing the algorithm, we will be using the OpenCV library in C++. OpenCV has a wide range of image processing tools and code in C++ should be easy to optimize and port to mobile platforms. We will develop the algorithm on a desktop computer, to allow for faster and easier testing, before we port the code to mobile. Our approach will be to test different algorithms on variety of hand images, starting with a naive approach and developing improved versions of the algorithm based on the results after each each iteration.