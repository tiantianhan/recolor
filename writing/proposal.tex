\documentclass[12pt, a4paper]{article}
\usepackage[a4paper, margin=1in]{geometry}

%paragraph spacing and indentation
\setlength{\parindent}{0em}
\setlength{\parskip}{1em}

\begin{document}

\section*{Design of mobile compatable image enhancement software for the believable transfer of hand skin color}
Tiantian Han, June 1, 2017

%general research gap, significance in society, current products and research
We often cannot judge based on the appearance of a personal apparel product on a model how that product will in actuality look on ourselves. Ideally, we would like the model demonstrating the product to be custom to us, to have the same skin colour and other bodily features as we do, so that we can better see how the product would look specifically on us. To address this issue there are already many applications in the beauty and fashion industry developed for the virtual trying on of products that either automatically applies the appearance of the product on an image of the user, or modifies the image of a model wearing the product to take on the appearance of the user.

For this project, we have opted for the latter approach. We have a video in a mobile app demonstrating nail polish on a model hand to allow virtual try-on of different nail polish colours. We would like to edit this demo video so that the model appears to have the user's exact skin colour, to help the user make an informed decision on whether a nail polish colour looks pleasant on their own hand. While it is possible to manually prepare a series of demo videos on models of different skin colours, preparing even a single video for virtual try-on is an extremely time intensive task. Moreover, each person has a particular skin colour and it's preferable to be able to tailor the video exactly to the skin colour of the user as the user is using the app.

To address these challenges, we propose developing an algorithm to automatically edit images of model hands to give them the user's skin colour. The resulting, edited images should be convincing and accurate to the provided skin colour. The algorithm should be able to quickly edit video frames on a mobile device such that the user notices no significant time lag to see the resulting demo video upon inputing their own skin colour. It should be straightforward for the user to provide their desired skin colour, possibly through providing an image of their own hand, although we will leave skin detection outside of the scope of this project and assume that the location of skin of the user's hand and the model hand are already specified an another process. Our goal is to develop an algorithm that would satisfy these requirements and evaluate its results.

Currently, we don't believe a satisfactory algorithm specific to our needs exists. While there are many research areas addressing

To develop the hand recoloring algorithm, our is to test different algorithms on variety of hand images, starting with a naive approach and iterating after visually evaluating each result to more complex algorithms. As sub-objectives, we would like to 

For developing and testing the algorithm, we will be using the OpenCV library in C++. OpenCV has a wide range of image processig tools and code in C++ should be easy to optimize and port to mobile apps. For this project we will develop the algorithm on a desktop computer, to allow for faster and easier testing, before we port the code to mobile.

% subgoals are to first develop an algorthm working for a set of hand images then optimize for mobile


%nail polish app, "specific" gap
% - mobile, realistic, requirement of beauty industry is subtlety
% In particular, for an application for demonstrating nail polish, we have opted for the latter approach. We would like to adjust the 

% specifically, list current research on skin colour transfer and note limitations - no mobile, no hand specific, not realistic

%objectives for software, goals and scope - 
% our research question or goal is wether we can develop a software will be suitable for smoothly performing transfer of hand on mobile
% software may have other uses - would be an exploration into the problem of realistic skin color transfer

% approach is to test different algorithms on variety of hand images and iterate after visually evaluating each result 
% subgoals are to first develop an algorthm working for a set of hand images then optimize for mobile

% methods will be to test the algorithm on desktop computer for faster, automated testing, but use programming language that would be easily ported to mobile

\pagebreak

\bibliographystyle{IEEEtran}
\bibliography{IEEEabrv,my_references}
\pagebreak

\end{document}