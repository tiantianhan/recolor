\documentclass[12pt, a4paper]{article}
\usepackage[a4paper, margin=1in]{geometry}

%paragraph spacing and indentation
\setlength{\parindent}{0em}
\setlength{\parskip}{1em}

\begin{document}

\section*{Design of mobile compatable image enhancement software for the believable transfer of hand skin color}
Tiantian Han, June 1, 2017

%general research gap, significance in society, current products and research
We often cannot judge based on the appearance of a personal apparel product on a model whether that product is suitable for us. Ideally, we would like the model to appear exactly like us, to have the same skin colour and bodily features, so that we can see how a product looks specifically on us. To address this issue there are already many applications in the beauty and fashion industry developed for product virtual try-on, whether to apply the appearance of the product on an image of the user, or to modify the image of a model wearing the product to take on the appearance of the user.

%nail polish app, "specific" gap
% - mobile, realistic, requirement of beauty industry is subtlety
In particular, for an application for demonstrating nail polish, we have opted for the latter approach. We would like to adjust the 

% specifically, list current research on skin colour transfer and note limitations - no mobile, no hand specific, not realistic

%objectives for software, goals and scope - 
% our research question or goal is wether we can develop a software will be suitable for smoothly performing transfer of hand on mobile
% software may have other uses - would be an exploration into the problem of realistic skin color transfer

% approach is to test different algorithms on variety of hand images and iterate after visually evaluating each result 
% subgoals are to first develope an algorthm working for a set of hand images then optimize for mobile

% methods will be to test the algorithm on desktop computer for faster, automated testing, but use programming language that would be easily ported to mobile

\pagebreak

\bibliographystyle{IEEEtran}
\bibliography{IEEEabrv,my_references}
\pagebreak

\end{document}