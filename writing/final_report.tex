\documentclass[12pt, a4paper]{article}
\usepackage{pdfpages}

\title{Software for believably adjusting the skin tone of a person's hand in an image}
\author{Tiantian Han}

\begin{document}
\includepdf{frontcover}

\section*{Abstract}
An Abstract that concisely summarizes the main points of your thesis work. The abstract should first frame your thesis work by identifying your research gap and approach, but should focus on the results and significance of your research or design work.

Up to 250 words, single-spaced in block form in centre of separate page. Paginate items 4-7 using roman numerals (i, ii, iii, iv) in the centre bottom of each page.
\pagebreak

\section*{Acknowledgements}
Acknowledge persons and/or supporting agencies in a single-spaced paragraph in block form in centre of separate page.
\pagebreak

%TODO
\section*{Table of Contents}
Begin on separate page, showing page numbers for each item, using the following conventions:
Chapters numbered: 1, 2, 3, etc.
Sections numbered: 1.1, 1.2, etc., 2.1, 2.2, etc. Appendices numbered: A, B, C, etc.
\pagebreak

\section*{List of Symbols, Figures, and Tables}
Include if relevant, and begin each respective list on a separate page. Figures, tables, equations, etc. should be numbered in a consistent manner.
\pagebreak

\section{Introduction}
An Introduction that concisely presents the context for the work that you are doing in your thesis, clearly identifies the research question/gap that your work is addressing, and identifies the central objectives. The framework for this component of the thesis has already been articulated twice; first, in your thesis proposal and second, in your interim report (IR). If the scope and nature of your work has not changed substantially, your introduction should be a revised version of your IR introduction, expanding on and clarifying components from that document where necessary. If your work has changed significantly, your introduction should reflect those changes, and describe context, gap, and goals related to your revised thesis project.
\pagebreak

\section{Literature Review}
The Literature Review (or Background) will explain any concept or theory that is important to your thesis (including prior, and current approaches to your problem) and summarize relevant research in the field to identify a gap in current knowledge. In essence, the literature review can be considered a more detailed, elaborated and well- supported version of the introduction. In the literature review, the gap is developed in significantly greater detail and supported by references to research. If your thesis work revolves around a design project, the literature review may help you to develop a deeper understanding of the design problem, and define and justify your requirements. As with item (1), the framework for this part of the final thesis document should already be in your thesis IR, and this section can be a revised and expanded version of the IR literature review.
\pagebreak

\section{Methods}
A Methods section that explains and justifies your experimental or design approach, your specific experimental methods or designs in a precise and detailed way that would allow someone else to capably reproduce them.
\pagebreak

\section{Results and Discussion}
A Results and Discussion section that presents your results, highlights key results, and interprets your results in the context of the research question established at the outset of the document. In a design based thesis, this section would provide an evaluation of the design based on established requirements.
\pagebreak

\section{Conclusion}
A brief Conclusion that summarizes the significance of your thesis work as a whole, and perhaps points to possible future research based on the groundwork laid in this thesis.

Identifies key claims to be drawn from results of research or design evaluation, qualifies them appropriately
       
Outlines significance of research done, identifies potential future work that arises from thesis work
\pagebreak

\section*{References}
Citations, in the body of the thesis, and references, at the end of the thesis body must be provided, properly formatted to a standard appropriate for your discipline (such as IEEE or CSE).
\pagebreak

\section*{Appendices}
Begin each appendix on a separate page, and number appendices A, B, C, etc.
\pagebreak

\end{document}