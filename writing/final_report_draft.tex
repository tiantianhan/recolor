\documentclass[12pt, a4paper]{article}
\usepackage[a4paper, margin=1in]{geometry} %adjust margins
\usepackage{cite} %citation formatting
\usepackage{array}
\usepackage{graphicx}
\usepackage{longtable}
\usepackage{etoolbox}
\usepackage{subcaption}
\usepackage{float}
\usepackage{amsmath}
\usepackage{mathtools}
\usepackage[nottoc]{tocbibind}
\usepackage[intoc]{nomencl} %nomenclature in table of contents

\usepackage{pdfpages} %including front page

\usepackage{setspace} %1.5 spacing
\setstretch{1.5}

%for clickable table of contents
\usepackage{hyperref}
\hypersetup{
    colorlinks,
    citecolor=black,
    filecolor=black,
    linkcolor=black,
    urlcolor=black
}

%define command for symbol denoting "average of"
\newcommand*\mean[1]{\bar{#1}}

%paragraph spacing and indentation
\setlength{\parindent}{0em}
\setlength{\parskip}{1em}

%counter for numbering the rows of tables
\newcounter{rowcntr}[table]
\renewcommand{\therowcntr}{\thetable.\arabic{rowcntr}}
% A new columntype to apply automatic stepping
\newcolumntype{N}{>{\refstepcounter{rowcntr}\therowcntr}c}
% Reset the rowcntr counter at each new tabular
\AtBeginEnvironment{tabular}{\setcounter{rowcntr}{0}}

\makenomenclature

\begin{document}

\includepdf{frontcover}

\renewcommand{\thepage}{\roman{page}}% Roman numerals for page counter
\tableofcontents
\pagebreak

\listoffigures
\listoftables
\pagebreak

\renewcommand{\nomname}{List of Symbols}
\printnomenclature
\pagebreak

\renewcommand{\thepage}{\arabic{page}}% Roman numerals for page counter

\section{Introduction}
%introduction
The modern beauty, cosmetics and apparel shopping experiences increasingly take place online and digitally, enhanced by \textit{virtual try-on} applications which demonstrate products on digital models to provide previews and vicarious experiences of the products for the users \cite{zhang_2017_try}. Increasingly these applications can either use images of the user themselves to demonstrate products or digitally modify the models to take on the user's physical characteristics such as skin colour, facial features and body measurements \cite{shilkrot_2013_garment, li_2015_replace}. A study by Merle et al. shows that this increasing ``perceived resemblance" between the user and the model increases the user's sense of connection to the model and is an important factor in increasing the user's perceived usefulness of the application and the positive user responses to the demonstrated products \cite{merle_2012_tryon}.

% The physical differences between ourselves and a model means that we often cannot judge based on the appearance of a product worn by a model how that product will actually look on ourselves. Ideally, to help us, the users, make better informed purchasing decisions, the model demonstrating the product should be customized to each user. To achieve this, there are already many applications developed for virtually trying on products in the beauty, cosmetics and garment industries which digitally modify the images of models to take on the appearance of the user \cite{zhang_2017_try} \cite{shilkrot_2013_garment, li_2015_replace}.
%better describe "virtual try-on?"

Our project is concerned with improving the perceived user-model resemblence for a particular virtual try-on mobile application, which currently demonstrates different nail polish colours on a video of a generic model hand. We would like instead to have the skin colour of the model hand be matched to each user's skin colour, so that the contrast between the nail polish colours and the user's skin can be clearly visualized. However, because preparing even a single video for virtual try-on is an extremely time intensive task, it is not feasible to manually prepare a large number of different demo videos with models of a range skin colours that would satisfy all users.

To address this challenge, we will develop an algorithm to incorporate into the application so that for each user using an instance of the application, the algorithm would quickly and automatically edit each frame of the video of the generic model hand so that the skin colour of the user is transfered to the image of the model hand. The user should only need to provide an image of their own hand as input, and a wide range of user skin colours should be supported by a single base video of a model of mid-toned skin colour. The process should be also able to run quickly on a mobile device, such that the user notices no significant time lag to see the resulting video upon inputting their own skin colour. We will discuss the requirements for our project in detail in Section \ref{sec:goals}.

%todo better describe colour transfer??
Currently, we aren't aware of an existing algorithm for skin colour transfer that satisfies all our specific requirements. While there has been a large body of work done addressing transfer of colour between images in general \cite{reinhard_2001_transfer, pitie_2005_pdf, chen_2014_propagation, chang_2015_palette, zhang_2017_decomposition}, only a smaller subset of the work addresses skin colour specifically\cite{yin_2004_transfer, seo_2005_transfer, yang_2017_semantic}. All such studies address images of facial portraits rather than hands, which often means that the bulk of the study is dedicated to solutions for the colour transfer of the various complex features of the face \cite{yang_2017_semantic}. Simple skin colour transfer is also used as a part of certain other imaging processing applications, but similarly, since the skin colour transfer is often only a small part of the whole project, the algorithms used are often relatively simple and not heavily designed for achieving accuracy to the user skin colour \cite{shilkrot_2013_garment, li_2015_replace}. In the related field of skin colour enhancement applications, the methods used are generally meant for small skin colour adjustments and not for making large changes to the user skin colour \cite{aradhye_2009_enhancement, lee_2010_mobile}. Finally, algorithms developed by most of the prior studies do not appear to be meant for use with the limited resources on a mobile device. We will discuss these previous studies of methods of skin colour transfer in detail in the following section, Section \ref{sec:academic_work}.

% Our goal is to develop a mobile compatible recolouring algorithm that would satisfy our requirements. As sub-objectives, we would like to first develop an effective algorithm and then optimize the algorithm's running time. We will focus solely on achieving convincing colour transfer in the algorithm, and assume that the location of the skin in the images are already determined by an another process. 

% For developing and testing the algorithm, we will be using the OpenCV library in C++. OpenCV has a wide range of image processing tools and code in C++ should be easy to optimize and port to mobile platforms. We will develop the algorithm on a desktop computer, to allow for faster and easier testing, before we port the code to mobile. Our approach will be to test different algorithms on variety of hand images, starting with a naive approach and developing improved versions of the algorithm based on the results after each each iteration.
\pagebreak

\section{Background and Literature Review}
%literature review
\section{Related prior work}

Relevant prior work fall into four rough categories. There is a large body of work on the subject of general automatic colour transfer and colour grading by example, transfering the ``style" or specific colours in an example image to another image. There have been several prior attempts at transfering specifically images wherein skin colour is prominent, and these we will discuss in detail. There are also several examples of practical application of skin transfer algorithms, where different application demonstrate practical uses of usually relatively simple skin transfer algorithm that is part of a larger project; we will discuss several of these projects. Finally, there is the field of skintone enhancement software, where algorithms are usually intended to adjust the user skin colour towards a more pleasing tone and not to a specific target colour. We include the latter because unlike the other categories of prior work there are several studies of adjusting skintone on a mobile device, which is part of the requirements for this project.

\section{Colour transfer by example image for general images}
Colour transfer refers to modifying the colours of an image to give it the desired appearance and style demonstrated by an example image, which we will refer to as the target image. Figure \ref{fig:color_transfer} illustrates an example of this effect.

There have been a wide range of work done in this area beginning with the seminal work of Reinhard et al. in 2001 \cite{reinhard_2001_transfer}. 

While these techniques are interesting possibilities to try when transfering human skin colour, because the these prior studies are all concerned with different problems that can arise with general images but not specifically for human skin colour, studies that specifically relate to human skin colour demonstrate that the general colour transfer techniques can be improved upon.

\section{Transfer of human skin colour}
Several studies have been done specifically on the transfer of human skin colour.

Seo et al. \cite{seo_2005_transfer} has a purpose closest to the purpose of this project, to transfer human skin colours. The authors show results that improve in realistic appearance compared to the Reinhard's algorithm.

%models the skin as ellipsoid distribution around vector, with another vector for specular reflection
%algorithm used - RGB space transform, division into bins and moving standard dev and mean while leaving details intact
%Results don't show range but we can possibly replicate and check?

It is not clear how fast the algorithm can run particularly on a mobile device, nor the range of colours that the algorithm can transform a single skin colour, and it is in these areas that our project will attempt to improve upon.

Yang et al. Performed the most recent study 


%explain theoretical concepts in context of thesis work; be clear and concise
%summarize relevant research to give understanding of current field
%analyze research in field to give deeper understanding of research question 
%indicate a path going forward



%photoshop
\subsection{Changing and matching skin colour in Photoshop}

Skin colour correction is a frequent problem encountered in photo retouching and there are a wide range of online video tutorials available documenting the methods artists use to manually adjust human skin tone in individual images using Adobe Photoshop, a widely used commercial image manipulation software. The purposes of these videos include giving the subject of an image the appearance of a tan, matching the skin tone of the subject to a desired skin tone on another individual, or matching the skin tone of a subject's face to the rest of the subject's body, which is often a slightly different colour \cite{photoshop:tan, photoshop:match_other, photoshop:match_body}. Bearing in mind that techniques described by such tutorials expect artistic input from a human editor to achieve the results and are therefore not entirely aligned with the purposes of this project, it is useful to study these methods because the results achieved are usually extremely realistic and aesthetically pleasing and should be a standard that the algorithm developed in this project strives towards. We therefore surveyed a number of these videos and summarize below the techniques of some of the most relevant tutorials.

\subsubsection*{Summary of Photoshop techniques}

Shaver demonstrated how to change a person's skin colour from dark to light \cite{photoshop:obama}. Shaver used levels and curves, which are tools that manipulate the $rgb$ colour histogram of the image, to increase brightness to an extent, then performed further brightening by using a grey scale conversion to brighten the skin area of a black and white image and then using the luminosity blend mode to place the colour back into the image. We show the results achieved in Table \ref{tab:obama_demo}.

\begin{table}[H]
    \centering
    \caption{Screen captures from Photoshop tutorial for changing skin colour from dark to light. \label{tab:obama_demo}}
\begin{tabular}{|c|c|}
    \hline
    Source & Output \\
    \hline
  \begin{minipage}{.29\textwidth}
    \includegraphics[width=\textwidth,height=\textheight,keepaspectratio]{images/obama_orig}
  \end{minipage} & 
  \begin{minipage}{.29\textwidth}
    \includegraphics[width=\textwidth,height=\textheight,keepaspectratio]{images/obama_res}
  \end{minipage} \\
    \hline
\end{tabular}
\end{table}

Phlearn demonstrated an effect in the reverse direction by demonstrating a technique for giving the model the appearance of a dark tan \cite{photoshop:tan}. The highlights and shadows of the image are adjusted separately by using the ``blend if" function of Photoshop, which blends in an effect only if the original pixel is above or below a certain threshold of brightness.

Phlearn also demonstrated a method for matching the skin colour of body and face in an image where the two appear mismatched \cite{photoshop:match_body}. The author sampled a range of colours from the body and adjusted the face with the levels tool for each colour channel. We show the results achieved in Table \ref{tab:match_body_demo}.

\begin{table}[H]
    \centering
    \caption{Screen captures from Photoshop tutorial for matching the skin tones of face and body. \label{tab:match_body_demo}}
\begin{tabular}{|c|c|c|}
    \hline
    Source & Target & Output \\
    \hline
  \begin{minipage}{.29\textwidth}
    \includegraphics[width=\textwidth,height=\textheight,keepaspectratio]{images/match_body_orig}
  \end{minipage} & 
  \begin{minipage}{.29\textwidth}
    \includegraphics[width=\textwidth,height=\textheight,keepaspectratio]{images/match_body_targ}
  \end{minipage} & 
  \begin{minipage}{.29\textwidth}
    \includegraphics[width=\textwidth,height=\textheight,keepaspectratio]{images/match_body_res}
  \end{minipage} \\
    \hline
\end{tabular}
\end{table}

PiXimperfect demonstrated a method for matching skin colour in one portrait to another \cite{photoshop:match_other}. PiXimperfect first calculates the two average colours of the faces and uses the Photoshop curves tool to match the average colours of the original image to the target image. There must then be further adjustments by eye to change colour, brightness and contrast. Examples of the results from PiXimperfect is show in Table \ref{tab:match_other_demo}

\begin{table}[H]
    \centering
    \caption{Screen captures from Photoshop tutorial for matching the skin tones of portraits of different people. \label{tab:match_other_demo}}
\begin{tabular}{|N||c|c|c|}
    \hline
    \multicolumn{1}{|c||}{No.} & Original & Target & Result \\
    \hline  \label{row:photoshop_match_other_1} &
  \begin{minipage}{.29\textwidth}
    \includegraphics[width=\textwidth,height=\textheight,keepaspectratio]{images/match_other_1_orig}
  \end{minipage} & 
  \begin{minipage}{.29\textwidth}
    \includegraphics[width=\textwidth,height=\textheight,keepaspectratio]{images/match_other_1_targ}
  \end{minipage} & 
  \begin{minipage}{.29\textwidth}
    \includegraphics[width=\textwidth,height=\textheight,keepaspectratio]{images/match_other_1_res}
  \end{minipage} \\
    \hline  \label{row:photoshop_match_other_2} &
  \begin{minipage}{.29\textwidth}
    \includegraphics[width=\textwidth,height=\textheight,keepaspectratio]{images/match_other_2_orig}
  \end{minipage} & 
  \begin{minipage}{.29\textwidth}
    \includegraphics[width=\textwidth,height=\textheight,keepaspectratio]{images/match_other_2_targ}
  \end{minipage} & 
  \begin{minipage}{.29\textwidth}
    \includegraphics[width=\textwidth,height=\textheight,keepaspectratio]{images/match_other_2_res}
  \end{minipage} \\
    \hline
\end{tabular}
\end{table}

Generally, for most of the techniques surveyed, levels and curves are used for small brightness adjustments \cite{photoshop:obama, photoshop:match_body, photoshop:match_other}, and often to reduce the vividness of the colour adjustments the saturation must be slightly decreased \cite{photoshop:obama, photoshop:match_body}. After all other effects are applied, the opacity of the overall effect is often reduced from 100\% for a more natural appearance \cite{photoshop:obama, photoshop:match_body}.

\subsubsection*{Limitations of Photoshop techniques}
The Photoshop techniques surveyed are not meant for automation; instead, they are meant to be tailored to each specific image that a human is adjusting, and there are many junctures where the specific numerical amount of an adjustment have to be judged by eye. While Photoshop has a method for automating processes to an extent using \textit{actions}, the processes are meant for increasing ease of use by artists who can make additional adjustments and are familiar with the tool, rather than for use in commercial applications where the process is entirely automated \cite{photoshop:actions}.

Another limitation is that Photoshop operates at a higher level of abstraction than image processing software making use of libraries such as OpenCV. Image processing code has much more control over processes that can be applied to images, and the regions on the image that processes are applied to. 

Finally, some Photoshop effects may be proprietary and are of course limited to the platforms that Photoshop supports, while a program developed with a platform such as OpenCV can be made open source and adapted to uses on a variety of different platforms.

\pagebreak

\section{Hand recolouration algorithms}
%methods 
We chose to write our algorithms in C++ using OpenCV libraries. Since C++ and OpenCV are available on both Android and iOS platforms as well, the code should be relatively easy to port to those platforms. To allow for faster testing of our code, we develop the code on OS X.

Eclipse is used to compile each iteration of the algorithm into a debug-mode executable program named Recolor. For ease of testing, as the algorithm is modified, we add more functionality to the Recolor program and retain the ability to use previous versions of the algorithm. We use a custom Python script to run new versions of Recolor from the terminal to test it. All of the relevant code and its versions are hosted on a git repository at github.com/tiantianhan/recolor

Recolor takes as input a hand image, a mask instructing it where to find the average skin colour of the hand, and a desired target skin colour. (Other flags and inputs are also used for testing purposes, see the Github repository README file for a full description of the usage.) Recolor then outputs the processed image where the skin tone is adjusted to the target colour.

We iterated from simple to more complex algorithms, at each step testing the algorithm and evaluating the results. We tested progressive iterations on a set of hand images with varying skin tones. The images are are shown in Figure \ref{img:input_hands_1}.

%all images that are not test results will be copied to the images folder

\begin{figure}[H]
    \centering
    \begin{subfigure}[b]{0.20\textwidth}
        \includegraphics[width=\textwidth]{images/hand_dark}
        \caption{}\label{img:input_hands_1_dark}
    \end{subfigure}
    ~
    \begin{subfigure}[b]{0.20\textwidth}
        \includegraphics[width=\textwidth]{images/hand_brown}
        \caption{}\label{img:input_hands_1_brown}
    \end{subfigure}
    ~
    \begin{subfigure}[b]{0.20\textwidth}
        \includegraphics[width=\textwidth]{images/hand_light}
        \caption{}\label{img:input_hands_1_light}
    \end{subfigure}
    ~
    \begin{subfigure}[b]{0.20\textwidth}
        \includegraphics[width=\textwidth]{images/hand_pale}
        \caption{}\label{img:input_hands_1_pale}
    \end{subfigure}
    \caption{Different hand images used for testing}\label{img:input_hands_1}
\end{figure}

 For each test, we called the Recolor program to transform the image of one hand to have the skin tone of the hand in another image, then visually compared the processed image to the image of the target hand. We performed the process on all possible combinations of our test images, paying particular attention to the extreme cases, such as transforming from Figure \ref{img:input_hands_1_dark} to Figure \ref{img:input_hands_1_pale} and vice versa, as well as cases that start with a hand with mid-tone skin such as in Figure \ref{img:input_hands_1_brown} (as this is the most likely use case for applications that change the image of a model's hand to match a range of skin tones). We evaluated the resulting images subjectively, based on whether the processed hand looks believably like a hand naturally of that skin tone, and noted any flaws that we then attempted to correct with the next iteration of the algorithm.

In the following subsections we summarize the results of each algorithm and our evaluation of the results.

\subsection{Naive approach: simple $RGB$ colour addition}
%boost algo
\subsubsection*{Algorithm}
To begin, we performed a simple addition of a value to each of the $rgb$ channels of the hand, such that the average colour of the hand in the processed image is equal to the average colour of the hand in the target image. The algorithm is shown in Equation \ref{eq:boost_algo}.

\begin{equation} \label{eq:boost_algo}
r' = r + \delta_r
\end{equation}

Where 

\begin{equation*}
\delta_r = \mean{r_t} - \mean{r}
\end{equation*}

\nomenclature{$r$}{Value of the red channel of a pixel in the original hand image.}
\nomenclature{$r'$}{Value of the red channel of the pixel after an algorithm is applied.}
\nomenclature{$\mean{r}$}{Average value of the red channel of the original hand image.}
\nomenclature{$\mean{r_t}$}{Average value of the red channel of the target hand image.}
\nomenclature{$g$}{Value of the green channel of a pixel in the original hand image.}
\nomenclature{$b$}{Value of the blue channel of a pixel in the original hand image.}

With the same equation applying for the $g$ and $b$ channels.

\subsubsection*{Results}
The complete results are shown in Table \ref{tab:boost_test} in Appendix \ref{app:boost}, a portion is shown here for convenience.

\begin{longtable}{|c||c|c|c|}
    \caption*{Portion of test results of simple addition / subtraction brightening function from Table \ref{tab:boost_test} in the Appendix \ref{app:boost}}\\
    \hline
    No. & Original & Target & Results \\ 
      \hline  \ref{row:boost_test_hand_dark_to_hand_light} &
  \begin{minipage}{.29\textwidth}
    \includegraphics[width=\textwidth,height=\textheight,keepaspectratio]{../inputs/hand_dark.jpg}
  \end{minipage} & 
  \begin{minipage}{.29\textwidth}
    \includegraphics[width=\textwidth,height=\textheight,keepaspectratio]{../inputs/hand_light.jpg}
  \end{minipage} & 
  \begin{minipage}{.29\textwidth}
    \includegraphics[width=\textwidth,height=\textheight,keepaspectratio]{../rc_test/outputs/20170516_boost_test/hand_dark_to_hand_light.jpg}
  \end{minipage} \\
    \hline  \ref{row:boost_test_hand_brown_to_hand_dark} &
  \begin{minipage}{.29\textwidth}
    \includegraphics[width=\textwidth,height=\textheight,keepaspectratio]{../inputs/hand_brown.jpg}
  \end{minipage} & 
  \begin{minipage}{.29\textwidth}
    \includegraphics[width=\textwidth,height=\textheight,keepaspectratio]{../inputs/hand_dark.jpg}
  \end{minipage} & 
  \begin{minipage}{.29\textwidth}
    \includegraphics[width=\textwidth,height=\textheight,keepaspectratio]{../rc_test/outputs/20170516_boost_test/hand_brown_to_hand_dark.jpg}
  \end{minipage} \\
\hline  \ref{row:boost_test_hand_brown_to_hand_light} &
  \begin{minipage}{.29\textwidth}
    \includegraphics[width=\textwidth,height=\textheight,keepaspectratio]{../inputs/hand_brown.jpg}
  \end{minipage} & 
  \begin{minipage}{.29\textwidth}
    \includegraphics[width=\textwidth,height=\textheight,keepaspectratio]{../inputs/hand_light.jpg}
  \end{minipage} & 
  \begin{minipage}{.29\textwidth}
    \includegraphics[width=\textwidth,height=\textheight,keepaspectratio]{../rc_test/outputs/20170516_boost_test/hand_brown_to_hand_light.jpg}
  \end{minipage} \\
  \hline  \ref{row:boost_test_hand_brown_to_hand_pale} &
  \begin{minipage}{.29\textwidth}
    \includegraphics[width=\textwidth,height=\textheight,keepaspectratio]{../inputs/hand_brown.jpg}
  \end{minipage} & 
  \begin{minipage}{.29\textwidth}
    \includegraphics[width=\textwidth,height=\textheight,keepaspectratio]{../inputs/hand_pale.jpg}
  \end{minipage} & 
  \begin{minipage}{.29\textwidth}
    \includegraphics[width=\textwidth,height=\textheight,keepaspectratio]{../rc_test/outputs/20170516_boost_test/hand_brown_to_hand_pale.jpg}
  \end{minipage} \\
    \hline
\end{longtable}

\subsubsection*{Evaluation}
Images of darker skin tones and smaller changes between the original skin tone and target colour to begin with (Row \ref{row:boost_test_hand_brown_to_hand_dark}) tend to have better results than images with large changes, especially towards lighter colours. This is likely because large changes force bright points in the original image to be truncated at white, and also causes dark regions on the image, such as shadows and grooves, to become significantly brighter and less close to true black, giving the image a ``high-key" look (Row \ref{row:boost_test_hand_dark_to_hand_light} and \ref{row:boost_test_hand_brown_to_hand_light}).

In addition, we noted that at this stage the transformation from a dark coloured hand to a very pale hand, or even from a mid-toned hand to a pale hand and vice versa is especially unconvincing. (Row \ref{row:boost_test_hand_brown_to_hand_pale}, also see \ref{row:boost_test_hand_dark_to_hand_pale} and \ref{row:boost_test_hand_pale_to_hand_dark})


\subsection{Proportional adjustment relative to average colour} \label{sec:algo_prop_eval}
%prop algorithm methods
\subsubsection*{Algorithm}
To correct for the effect of the bright spots in the image being over bright and the high-key appearance resulting from all the shadows being brightened, we used an algorithm that maps the black and white points of the image to the same value, and adjusts the colours in between to match the target average colour. The algorithm is shown in Equation \ref{eq:prop_algo}.

\begin{equation} \label{eq:prop_algo}
  r' = \left.
  \begin{dcases}
    \displaystyle \Big(\frac{\mean{r_t}}{\mean{r}}\Big)r, & \text{for } r \leq \mean{r} \\
    \displaystyle 255 - 
    \Big(\frac{255 - \mean{r_t}}{255 - \mean{r}}\Big)(255 - r), & \text{for } r > \mean{r} \\
  \end{dcases}
  \right.
\end{equation}



With the same equation applying for the $g$ and $b$ channels.

\subsubsection*{Results}
The complete results are shown in Table \ref{tab:prop_test} in Appendix \ref{app:prop}, and a portion is shown here for convenience.

\begin{longtable}{|c||c|c|c|}
    \caption*{Portion of test results of adjusting proportionally based on distance of col or to the average from Table \ref{tab:prop_test} in the Appendix \ref{app:prop}}\\
    \hline
    No. & Original & Target & Results \\
    \hline  \ref{row:prop_test_hand_dark_to_hand_light} &
  \begin{minipage}{.29\textwidth}
    \includegraphics[width=\textwidth,height=\textheight,keepaspectratio]{../inputs/hand_dark.jpg}
  \end{minipage} & 
  \begin{minipage}{.29\textwidth}
    \includegraphics[width=\textwidth,height=\textheight,keepaspectratio]{../inputs/hand_light.jpg}
  \end{minipage} & 
  \begin{minipage}{.29\textwidth}
    \includegraphics[width=\textwidth,height=\textheight,keepaspectratio]{../rc_test/outputs/20170516_proportional_test/hand_dark_to_hand_light.jpg}
  \end{minipage} \\
    \hline  \ref{row:prop_test_hand_brown_to_hand_dark} &
  \begin{minipage}{.29\textwidth}
    \includegraphics[width=\textwidth,height=\textheight,keepaspectratio]{../inputs/hand_brown.jpg}
  \end{minipage} & 
  \begin{minipage}{.29\textwidth}
    \includegraphics[width=\textwidth,height=\textheight,keepaspectratio]{../inputs/hand_dark.jpg}
  \end{minipage} & 
  \begin{minipage}{.29\textwidth}
    \includegraphics[width=\textwidth,height=\textheight,keepaspectratio]{../rc_test/outputs/20170516_proportional_test/hand_brown_to_hand_dark.jpg}
  \end{minipage} \\
\hline  \ref{row:prop_test_hand_brown_to_hand_light} &
  \begin{minipage}{.29\textwidth}
    \includegraphics[width=\textwidth,height=\textheight,keepaspectratio]{../inputs/hand_brown.jpg}
  \end{minipage} & 
  \begin{minipage}{.29\textwidth}
    \includegraphics[width=\textwidth,height=\textheight,keepaspectratio]{../inputs/hand_light.jpg}
  \end{minipage} & 
  \begin{minipage}{.29\textwidth}
    \includegraphics[width=\textwidth,height=\textheight,keepaspectratio]{../rc_test/outputs/20170516_proportional_test/hand_brown_to_hand_light.jpg}
  \end{minipage} \\
  \hline  \ref{row:prop_test_hand_brown_to_hand_pale} &
  \begin{minipage}{.29\textwidth}
    \includegraphics[width=\textwidth,height=\textheight,keepaspectratio]{../inputs/hand_brown.jpg}
  \end{minipage} & 
  \begin{minipage}{.29\textwidth}
    \includegraphics[width=\textwidth,height=\textheight,keepaspectratio]{../inputs/hand_pale.jpg}
  \end{minipage} & 
  \begin{minipage}{.29\textwidth}
    \includegraphics[width=\textwidth,height=\textheight,keepaspectratio]{../rc_test/outputs/20170516_boost_test/hand_brown_to_hand_pale.jpg}
  \end{minipage} \\
    \hline
\end{longtable}

\subsubsection*{Evaluation}
This method improved the appearance of cases with over-bright spots or ``high-key" appearance issues, as Figure \ref{img:compare_bright_spot} shows:
\begin{figure}[H]
    \centering
    \begin{subfigure}[b]{0.40\textwidth}
        \includegraphics[width=\textwidth]{../rc_test/outputs/20170516_boost_test/hand_dark_to_hand_light.jpg}
        \caption{Simple addition algorithm (Equation \ref{eq:boost_algo})result}
    \end{subfigure}
    ~
    \begin{subfigure}[b]{0.40\textwidth}
        \includegraphics[width=\textwidth]{../rc_test/outputs/20170516_proportional_test/hand_dark_to_hand_light.jpg}
        \caption{Proportional adjustment algorithm (Equation \ref{eq:prop_algo}) result}
    \end{subfigure}
    \caption{Comparison of algorithm \ref{eq:boost_algo} and \ref{eq:prop_algo} results for transforming a dark hand (Figure \ref{img:input_hands_1_dark}) to a light hand (Figure \ref{img:input_hands_1_light}).\label{img:compare_bright_spot}}
\end{figure}

We noted however, that this method noticeably does not correct for, and even exacerbates slightly relative to the simple addition algorithm the dark spots at the joints and creases of a hand of darker skin tone when it is transformed to a lighter skin tone (Row \ref{row:prop_test_hand_brown_to_hand_light}). Other results are similar to the results of the simple addition algorithm.
 

\subsection{Correcting for dark spots}
%proportional corrected algorithm methods
\subsubsection*{Algorithm}
We attempted to correct the dark spot issue by significantly reducing the absolute difference between dark pixels and the average colour, ensuring that the dark spots would instead have colours close to the average. We perform this correction on the output of the proportional adjustment algorithm.

\begin{equation} \label{eq:prop_corr_algo}
  r'' = \left.
  \begin{dcases}
    \displaystyle \mean{r'} - \frac{(\mean{r'} - r')}{\alpha}, & \text{for } r' < \mean{r'} \\
    \displaystyle r', & \text{for } r' \geq \mean{r'} \\
  \end{dcases}
  \right.\\
\end{equation}

\nomenclature{$r''$}{Value of the red channel of the pixel after a second algorithm is applied.}
\nomenclature{$\mean{r'}$}{Average value of the red channel of the hand image outputed by the first algorithm.}

Where $\alpha$ is a constant, $\alpha  > 1$. The same equation applies for the $g$ and $b$ channels.

\subsubsection*{Results}
See Table \ref{tab:prop_correct_test} in Appendix \ref{app:prop_corr_a10}, \ref{app:prop_corr_a5} and \ref{app:prop_corr_a3} for complete results for a range of values for $\alpha$. A portion of the results for $\alpha = 3$ is reproduced here for convenience.

\begin{longtable}{|c||c|c|c|}
    \caption*{Test results of proportional adjusting with correction for dark spots, alpha = 3 from Table \ref{tab:prop_correct_test_a3} in the Appendix \ref{app:prop_corr_a3}}\\
    \hline
    No. & Original & Target & Results \\
    \hline  \ref{row:prop_correct_test_a3_hand_dark_to_hand_brown} &
  \begin{minipage}{.29\textwidth}
    \includegraphics[width=\textwidth,height=\textheight,keepaspectratio]{../inputs/hand_dark.jpg}
  \end{minipage} & 
  \begin{minipage}{.29\textwidth}
    \includegraphics[width=\textwidth,height=\textheight,keepaspectratio]{../inputs/hand_light.jpg}
  \end{minipage} & 
  \begin{minipage}{.29\textwidth}
    \includegraphics[width=\textwidth,height=\textheight,keepaspectratio]{../rc_test/outputs/20170517_proportional_corrected_test_alpha3/hand_dark_to_hand_light.jpg}
  \end{minipage} \\
    \hline  \ref{row:prop_correct_test_a3_hand_brown_to_hand_dark} &
  \begin{minipage}{.29\textwidth}
    \includegraphics[width=\textwidth,height=\textheight,keepaspectratio]{../inputs/hand_brown.jpg}
  \end{minipage} & 
  \begin{minipage}{.29\textwidth}
    \includegraphics[width=\textwidth,height=\textheight,keepaspectratio]{../inputs/hand_dark.jpg}
  \end{minipage} & 
  \begin{minipage}{.29\textwidth}
    \includegraphics[width=\textwidth,height=\textheight,keepaspectratio]{../rc_test/outputs/20170517_proportional_corrected_test_alpha3/hand_brown_to_hand_dark.jpg}
  \end{minipage} \\
\hline  \ref{row:prop_correct_test_a3_hand_brown_to_hand_light} &
  \begin{minipage}{.29\textwidth}
    \includegraphics[width=\textwidth,height=\textheight,keepaspectratio]{../inputs/hand_brown.jpg}
  \end{minipage} & 
  \begin{minipage}{.29\textwidth}
    \includegraphics[width=\textwidth,height=\textheight,keepaspectratio]{../inputs/hand_light.jpg}
  \end{minipage} & 
  \begin{minipage}{.29\textwidth}
    \includegraphics[width=\textwidth,height=\textheight,keepaspectratio]{../rc_test/outputs/20170517_proportional_corrected_test_alpha3/hand_brown_to_hand_light.jpg}
  \end{minipage} \\
  \hline  \ref{row:prop_correct_test_a3_hand_brown_to_hand_pale} &
  \begin{minipage}{.29\textwidth}
    \includegraphics[width=\textwidth,height=\textheight,keepaspectratio]{../inputs/hand_brown.jpg}
  \end{minipage} & 
  \begin{minipage}{.29\textwidth}
    \includegraphics[width=\textwidth,height=\textheight,keepaspectratio]{../inputs/hand_pale.jpg}
  \end{minipage} & 
  \begin{minipage}{.29\textwidth}
    \includegraphics[width=\textwidth,height=\textheight,keepaspectratio]{../rc_test/outputs/20170517_proportional_corrected_test_alpha3/hand_brown_to_hand_pale.jpg}
  \end{minipage} \\
    \hline
\end{longtable}

\subsubsection*{Evaluation}
As show in Figure \ref{img:compare_dark_spot}, the dark spots and creases noted in Section \ref{sec:algo_prop_eval} are reduced.

\begin{figure}[H]
    \centering
    \begin{subfigure}[b]{0.40\textwidth}
        \includegraphics[width=\textwidth]{../rc_test/outputs/20170516_proportional_test/hand_brown_to_hand_light.jpg}
        \caption{Proportional adjustment algorithm (Equation \ref{eq:prop_algo})result}
    \end{subfigure}
    ~
    \begin{subfigure}[b]{0.40\textwidth}
        \includegraphics[width=\textwidth]{../rc_test/outputs/20170517_proportional_corrected_test_alpha3/hand_brown_to_hand_light.jpg}
        \caption{Proportional adjustment algorithm with correction (Equation \ref{eq:prop_corr_algo}) result}
    \end{subfigure}
    \caption{Comparison of algorithms \ref{eq:prop_algo} and \ref{eq:prop_corr_algo} results for transforming a mid-toned hand (Figure \ref{img:input_hands_1_brown}) to a light hand (Figure \ref{img:input_hands_1_light}).\label{img:compare_dark_spot}}
\end{figure}

However, we have noted that even for the more lower strength effect where $\alpha = 3$, this process brings back the problem of eliminating true black from the image, making the image appear once again to be without shadows (See Row \ref{row:prop_correct_test_a3_hand_dark_to_hand_brown} and \ref{row:prop_correct_test_a3_hand_brown_to_hand_light}). 

Also, it should be noted that for completeness, we performed this algorithm on all possible combinations of hand images, even though it was targeted at cases where the original image is of a skin tone darker than the new image. 

Interestingly, in a couple of cases of transformation from lighter skin tone to darker skin tone, the hand takes on a darker colour around joints in a similar manner to how a hand of dark skin tone is naturally coloured, though the algorithm was not intended for this purpose (See Row \ref{row:prop_correct_test_a3_hand_brown_to_hand_dark} as well as \ref{row:prop_correct_test_a3_hand_light_to_hand_dark}, \ref{row:prop_correct_test_a3_hand_light_to_hand_brown}). This is not always the case, however. Row \ref{row:prop_correct_test_a3_hand_pale_to_hand_dark} does not seem to exhibit this effect, possibly due to the different lighting of the hand in the image.

\subsection{Ignoring shadows in calculating average colour}
%adding correction for percentage of skin
\subsubsection*{Algorithm}
We noticed that in the case where the target image has a pale hand and relatively dark shadows such as in Figure \ref{img:input_hands_1_light}, the average colour calculated for the target hand is too dark, causing the skin colour of the results to appear darker than the skin colour of the target. We correct this by calculating the average skin colour of the target hand with only a percentage of the brightest pixels in the original region of interest used to calculate the average. 

\begin{equation} \label{eq:prop_corr_ave_algo}
  %TODO add equation
\end{equation}

\subsubsection*{Results}
We generated the resultant images using several different values for the percentile of brightest pixels, and determined that the 10th percentile gave the best results, as shown in Table \ref{tab:prop_correct_test_a1p1_ave10}. In Appendices \ref{app:prop_corr_ave_a1p1_perc5} and \ref{app:prop_corr_ave_a1p1_perc25} we show the results for some other percentiles.

\begin{longtable}{|N||c|c|c|}
	\caption{Test results of proportional brightening with correction for dark spots and using brightest 10 percent of pixels to calculate average colour, $\alpha$ = 1.1\label{tab:prop_correct_test_a1p1_ave10}}\\
	\hline
	\multicolumn{1}{|c||}{No.} & Original & Target & Result \\ 
	\hline
	\input{latex_maker/prop_correct_ave_test_a1p1_perc10_label/hand_dark_to_hand_brown}
	\input{latex_maker/prop_correct_ave_test_a1p1_perc10_label/hand_dark_to_hand_light}
	\input{latex_maker/prop_correct_ave_test_a1p1_perc10_label/hand_dark_to_hand_pale}
	\input{latex_maker/prop_correct_ave_test_a1p1_perc10_label/hand_brown_to_hand_dark}
	\input{latex_maker/prop_correct_ave_test_a1p1_perc10_label/hand_brown_to_hand_light}
	\input{latex_maker/prop_correct_ave_test_a1p1_perc10_label/hand_brown_to_hand_pale}
	\input{latex_maker/prop_correct_ave_test_a1p1_perc10_label/hand_light_to_hand_dark}
	\input{latex_maker/prop_correct_ave_test_a1p1_perc10_label/hand_light_to_hand_brown}
	\input{latex_maker/prop_correct_ave_test_a1p1_perc10_label/hand_light_to_hand_pale}
	\input{latex_maker/prop_correct_ave_test_a1p1_perc10_label/hand_pale_to_hand_dark}
	\input{latex_maker/prop_correct_ave_test_a1p1_perc10_label/hand_pale_to_hand_brown}
	\input{latex_maker/prop_correct_ave_test_a1p1_perc10_label/hand_pale_to_hand_light}

 \end{longtable}

\subsubsection*{Evaluation}
Figure \ref{img:10_perc_mask} demonstrates the new regions used to calculate the average skin colour. We can see that the areas with shadows are effectively discarded, and the average colour calculated is significantly lighter and visually closer to the skin colour of the target hand. 

\begin{figure}[H]
\centering
\begin{tabular}{ccc}
    \multirow{2}{*}[5em]{\begin{subfigure}[b]{0.30\textwidth}
        \includegraphics[width=\textwidth]{images/hand_pale}
        \caption{Input hand image with significant shadows}\label{img:alg_3_eval_hand_pale}
    \end{subfigure}}&
    \begin{subfigure}[b]{0.30\textwidth}
        \includegraphics[width=\textwidth]{images/pale_ave_10_original_mask}
        \caption{Mask used to calculate color in Algorithm \ref{eq:prop_corr_algo}}\label{img:original_mask}
    \end{subfigure} &
    \begin{subfigure}[b]{0.30\textwidth}
        \includegraphics[width=\textwidth]{images/ave_col_100}
        \caption{Average color calculated with Algorithm \ref{eq:prop_corr_algo}}\label{img:ave_col_100}
    \end{subfigure}\\
    &
    \begin{subfigure}[b]{0.30\textwidth}
        \includegraphics[width=\textwidth]{images/pale_ave_10_adjusted_mask}
        \caption{Mask used to calculate color in Algorithm \ref{eq:prop_corr_ave_algo}}\label{img:adjusted_mask}
    \end{subfigure} &
    \begin{subfigure}[b]{0.30\textwidth}
        \includegraphics[width=\textwidth]{images/ave_col_10}
        \caption{Average color calculated with Algorithm \ref{eq:prop_corr_ave_algo}}\label{img:ave_col_10}
    \end{subfigure}
\end{tabular}
\caption{The average color calculation process in Algorithm \ref{eq:prop_corr_ave_algo} compared the previous version, Algorithm \ref{eq:prop_corr_algo}}\label{img:10_perc_mask}
\end{figure}
\pagebreak

\subsection{Summary and evaluation of the complete algorithm}
%Summary of the complete algorithm
The complete algorithm (comprised of Algorithms \ref{eq:prop_algo}, \ref{eq:prop_corr_algo} and \ref{eq:prop_corr_ave_algo}) is summarized in the flow chart in Figure \ref{img:algo_summary}.

\begin{figure}[H]
    \centering
    \includegraphics[width=\textwidth]{images/algo_diagram}
    \caption{Summary of complete algorithm for hand colour transfer}\label{img:algo_summary}
\end{figure}

In reference to our original goals and constraints listed in Section \ref{sec:goals}, we have made the most progress in the area of accurate and realistic transfer of skin colour from mid-toned and light coloured hands to other colours. However, the range of colours that we can realistic transfer between remains limited. For future work, we should further test our algorithm on a larger set of hand images to determine whether there are any other issues. Finally, we have yet to profile and optimize the code for performance on a mobile devices; however, since we have written the code using a language and libraries that should be easy to port to a mobile platform, we believe that optimizing and porting the code should be a very feasible task for future work.
\pagebreak

\section{Future Work}
As a next step, we would like to improve our algorithm so that the results for transforming between large colour differences are more convincing. A possible approach for the case of transforming to the very pale hand in Figure \ref{img:input_hands_1_pale} in particular is to obtain a better average colour from the target hand by ignoring the shadows in the image. We would also like to compare our results with some of the other algorithms for colour transfer that we have found through the background research.
\pagebreak

\bibliographystyle{IEEEtran}
\bibliography{IEEEabrv,my_references}
\pagebreak

\appendix
\section{Results for proportional adjustment with darkspot correction, $\alpha = 1.5$}\label{app:prop_corr_ave_a1p5}
\input{latex_maker/prop_correct_test_a1p5-summary}

\section{Results for proportional adjustment with darkspot correction, $\alpha = 5$}\label{app:prop_corr_ave_a5}
\input{latex_maker/prop_correct_test_a5-summary}

% \section{Complete results for proportional adjustment with darkspot correction, $\alpha = 1.1$, calculating target average color with 5th percentile bright pixels}\label{app:prop_corr_ave_a1p1_perc5}
% \begin{longtable}{|N||c|c|c|}
	\caption{}\\
	\hline
	\multicolumn{1}{|c||}{No.} & Original & Target & Result \\ 
	\hline
	\input{latex_maker/prop_correct_ave_test_a1p1_perc5_label/hand_dark_to_hand_brown}
	  \label{row:PY_NAME_hand_dark_to_hand_light} &
  \begin{minipage}{.29\textwidth}
    \includegraphics[width=\textwidth,height=\textheight,keepaspectratio]{../inputs/hand_dark.jpg}
  \end{minipage} & 
  \begin{minipage}{.29\textwidth}
    \includegraphics[width=\textwidth,height=\textheight,keepaspectratio]{../inputs/hand_light.jpg}
  \end{minipage} & 
  \begin{minipage}{.29\textwidth}
    \includegraphics[width=\textwidth,height=\textheight,keepaspectratio]{../rc_test/outputs/20170524_prop_corr_1p1_ave_100/hand_dark_to_hand_light.jpg}
  \end{minipage} \\
\hline
	\input{latex_maker/prop_correct_ave_test_a1p1_perc5_label/hand_dark_to_hand_pale}
	\input{latex_maker/prop_correct_ave_test_a1p1_perc5_label/hand_brown_to_hand_dark}
	\input{latex_maker/prop_correct_ave_test_a1p1_perc5_label/hand_brown_to_hand_light}
	\input{latex_maker/prop_correct_ave_test_a1p1_perc5_label/hand_brown_to_hand_pale}
	\input{latex_maker/prop_correct_ave_test_a1p1_perc5_label/hand_light_to_hand_dark}
	\input{latex_maker/prop_correct_ave_test_a1p1_perc5_label/hand_light_to_hand_brown}
	  \label{row:PY_NAME_hand_light_to_hand_pale} &
  \begin{minipage}{.29\textwidth}
    \includegraphics[width=\textwidth,height=\textheight,keepaspectratio]{../inputs/hand_light.jpg}
  \end{minipage} & 
  \begin{minipage}{.29\textwidth}
    \includegraphics[width=\textwidth,height=\textheight,keepaspectratio]{../inputs/hand_pale.jpg}
  \end{minipage} & 
  \begin{minipage}{.29\textwidth}
    \includegraphics[width=\textwidth,height=\textheight,keepaspectratio]{../rc_test/outputs/20170524_prop_corr_1p1_ave_100/hand_light_to_hand_pale.jpg}
  \end{minipage} \\
\hline
	  \ref{row:PY_NAME_hand_pale_to_hand_dark} &
  \begin{minipage}{.29\textwidth}
    \includegraphics[width=\textwidth,height=\textheight,keepaspectratio]{../inputs/hand_pale.jpg}
  \end{minipage} & 
  \begin{minipage}{.29\textwidth}
    \includegraphics[width=\textwidth,height=\textheight,keepaspectratio]{../inputs/hand_dark.jpg}
  \end{minipage} & 
  \begin{minipage}{.29\textwidth}
    \includegraphics[width=\textwidth,height=\textheight,keepaspectratio]{../rc_test/outputs/20170524_prop_corr_1p1_ave_25/hand_pale_to_hand_dark.jpg}
  \end{minipage} \\
\hline
	\input{latex_maker/prop_correct_ave_test_a1p1_perc5_label/hand_pale_to_hand_brown}
	  \label{row:PY_NAME_hand_pale_to_hand_light} &
  \begin{minipage}{.29\textwidth}
    \includegraphics[width=\textwidth,height=\textheight,keepaspectratio]{../inputs/hand_pale.jpg}
  \end{minipage} & 
  \begin{minipage}{.29\textwidth}
    \includegraphics[width=\textwidth,height=\textheight,keepaspectratio]{../inputs/hand_light.jpg}
  \end{minipage} & 
  \begin{minipage}{.29\textwidth}
    \includegraphics[width=\textwidth,height=\textheight,keepaspectratio]{../rc_test/outputs/20170524_prop_corr_1p1_ave_10/hand_pale_to_hand_light.jpg}
  \end{minipage} \\
\hline

 \end{longtable}

% \section{Complete results for proportional adjustment with darkspot correction, $\alpha = 1.1$, calculating target average color with 25th percentile bright pixels}\label{app:prop_corr_ave_a1p1_perc25}
% \input{latex_maker/prop_correct_ave_test_a1p1_perc25-label}

% \section{Complete results for proportional adjustment with darkspot correction, $\alpha = 1.1$, calculating target average color with 100th percentile bright pixels}\label{app:prop_corr_ave_a1p1_perc100}
% \input{latex_maker/prop_correct_ave_test_a1p1_perc100-label}

\end{document}